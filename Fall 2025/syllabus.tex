\documentclass{article}
\usepackage{graphicx, float} % Required for inserting images
\usepackage[letterpaper, margin=1in]{geometry}
\usepackage{amsfonts, amsmath, hyperref, listings, multirow, physics, placeins, thmtools}

\title{Computational Modeling in Engineering and the Sciences}
\author{Ashton Cole \thanks{Graduate Research Assistant, Computational Hydraulics Group \\ Oden Institute for Computational Engineering \& Sciences \\ \href{mailto:ashtonvcole@utexas.edu}{ashtonvcole@utexas.edu}}}
\date{Fall 2025}

\begin{document}
	\maketitle
	
	\section*{Description}
	In this beginner-friendly group, we will broadly explore the field of computational modeling: the kind of work done at the Oden Institute (\href{https://oden.utexas.edu}{https://oden.utexas.edu}). We will practice both reading papers and coding simple projects. Each meeting will consist of a discussion, lecture, and self-guided coding demo. Relevant literature will be assigned for reading between each session.
	
	We will start with learning how to build and solve ordinary differential equations, then partial differential equations. Emphasis will be on applications in storm surge modeling, my area of research, but other groups and topics will be considered on the way. The semester will culminate in a mini-project solving the nonlinear Shallow Water Equations.
	
	\section*{Goals}
	\begin{itemize}
		\item \textbf{Learn about computational modeling.} We will explore at the surface level common modeling techniques and solution schemes used by engineers and scientists to model real-world problems.
		\item \textbf{Explore research and academia.} We will explore the work at the Oden Institute and resources available to students like the Texas Advanced Computing Center (\href{https://tacc.utexas.edu}{https://tacc.utexas.edu}).
		\item \textbf{Mentorship.} I am happy to be a resource for anyone interested in research and academia, whether that means an undergraduate job for a couple of semesters, or aspirations for professorships and tenure.
		\item \textbf{Build up math and coding skills.} Although participants are not expected to pursue hydraulics or even computational science, the techniques covered are broadly applicable. They should be useful exposure for any engineer, mathematician, or computer scientist.
		\begin{itemize}
			\item Math skills: ODEs, PDEs, numerical solution techniques
			\item Programming skills: Python, Jupyter, Conda, NumPy, control sequences, data structures, vectorization, visualization, file I/O
		\end{itemize}
		\item \textbf{``Leave things better than we found them."} If anything, this is a chance to try new things, discover your interests, and grow as a person.
	\end{itemize}
	
	\section*{Prerequisites}
	\begin{itemize}
		\item Required
		\begin{itemize}
			\item Programming fundamentals
			\item Calculus
			\item Understanding of basic matrix operations
			\item Willingness to learn new things
		\end{itemize}
		\item Recommended
		\begin{itemize}
			\item Python
			\item Vector calculus
			\item Linear algebra
			\item Differential equations
			\item Interest in domain outside of computer science, e.g. physics
		\end{itemize}
	\end{itemize}

	\section*{Schedule}
	A tentative schedule is provided below, subject to change based on need and mentee feedback. Fewer sessions than weeks are scheduled in anticipation of conflicts and end-of-semester chaos.
	
	\begin{tabular}{l|llp{4cm}p{4cm}}
		& Date & Paper & Lecture & Demo \\
		\hline
		1 &  & N/A & Overview of DiRP & Setup of Python, conda, Jupyter, NumPy \\
		\hline
		2 &  & Something about COVID & Scalar ODEs, integration techniques, error analysis & Population dynamics \\
		\hline
		3 &  & Something from Stella & General ODEs & Computational chemistry \\
		\hline
		4 &  & Something from Moriba Jah & Formulating two-body problem & Astrodynamics \\
		\hline
		5 &  & Something HPC & HPC and TACC & TACC ``tour'' \\
		\hline
		6 &  & Something quantum & PDEs and solution techniques & Scalar advection equation \\
		\hline
		7 &  & Something from cardiology or oncology & Transport equations & Fun PDEs I \\
		\hline
		8 &  & Something by Dawson & Computational hydraulics and Shallow Water Equations & Fun PDEs II \\
		\hline
		9 &  & Something by Dawson & Constructing a simulation & SWE I \\
		\hline
		10 &  & Something by Dawson & Postprocessing & SWE II \\
	\end{tabular}
	
	special days
	acoustics and waves
	pdes and transport??
	TACC
\end{document}